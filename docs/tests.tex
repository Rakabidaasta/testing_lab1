\documentclass{article}
\usepackage[T2A]{fontenc}
\usepackage[utf8]{inputenc}
\usepackage[english, russian]{babel}

\begin{document}
\section{Позитивные тесты}
\subsection{one\_solve -- уравнения с одим решением}
\subsubsection{suite1}
Входные данные -- $a=1, b=0, c=0$\\
Ожидаемый результат: структура solves:
\begin{itemize}
    \item solves.count == 1
    \item solves.x1 == 0.0 (с точностью 0.000001)
\end{itemize}
\subsubsection{suite2}
Входные данные -- $a=1, b=2, c=1$\\
Ожидаемый результат: структура solves:
\begin{itemize}
    \item solves.count == 1
    \item solves.x1 == -1.0 (с точностью 0.000001)
\end{itemize}
\subsubsection{suite3}
Входные данные -- $a=1, b=-4, c=4$\\
Ожидаемый результат: структура solves:
\begin{itemize}
    \item solves.count == 1
    \item solves.x1 == 2.0 (с точностью 0.000001)
\end{itemize}

\subsection{two\_solves -- уравнения с двумя решениями}
\subsubsection{suite1}
Входные данные -- $a=1, b=-5, c=6$\\
Ожидаемый результат: структура solves:
\begin{itemize}
    \item solves.count == 2
    \item solves.x1 == 2.0 (с точностью 0.000001)
    \item solves.x2 == 3.0 (с точностью 0.000001)
\end{itemize}
\subsubsection{suite2}
Входные данные -- $a=2, b=2, c=-4$\\
Ожидаемый результат: структура solves:
\begin{itemize}
    \item solves.count == 2
    \item solves.x1 == -2.0 (с точностью 0.000001)
    \item solves.x2 == 1.0 (с точностью 0.000001)
\end{itemize}
\subsubsection{suite3}
Входные данные -- $a=1, b=5, c=6$\\
Ожидаемый результат: структура solves:
\begin{itemize}
    \item solves.count == 2
    \item solves.x1 == -3.0 (с точностью 0.000001)
    \item solves.x2 == -2.0 (с точностью 0.000001)
\end{itemize}

\subsection{no\_solves -- уравнения без решений}
\subsubsection{suite1}
Входные данные -- $a=1, b=2, c=3$\\
Ожидаемый результат: структура solves:
\begin{itemize}
    \item solves.count == 0
\end{itemize}
\subsubsection{suite2}
Входные данные -- $a=6, b=6, c=9$\\
Ожидаемый результат: структура solves:
\begin{itemize}
    \item solves.count == 0
\end{itemize}
\subsubsection{suite3}
Входные данные -- $a=-6, b=-6, c=-9$\\
Ожидаемый результат: структура solves:
\begin{itemize}
    \item solves.count == 0
\end{itemize}

\section{Негативные тесты}
\subsection{extra\_tests -- дополнительные тесты. Проверяют работу функции при условии, что параметры <<a>>, <<b>> или <<c>> равны 0}
\subsubsection{linear\_equation -- линейное уравнение}
Входные данные -- $a=0, b=1, c=2$\\
Ожидаемый результат: структура solves:
\begin{itemize}
    \item solves.count == 1
    \item solves.x1 == -2.0 (с точностью 0.000001)
\end{itemize}
\subsubsection{zero -- нет решений}
Входные данные -- $a=0, b=0, c=1$\\
Ожидаемый результат: структура solves:
\begin{itemize}
    \item solves.count == 0
\end{itemize}
\subsubsection{infinity -- бесконечно много решений}
Входные данные -- $a=0, b=0, c=0$\\
Ожидаемый результат: структура solves:
\begin{itemize}
    \item solves.count == 0
\end{itemize}

\end{document}